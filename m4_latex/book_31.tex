\documentclass[12pt,a4paper]{article}
\usepackage[utf8]{inputenc}

\title{Classical Mechanical}
\author{Santolas de Sousa}
\date{December 2016}

\usepackage{graphicx}

\begin{document}

\begin{titlepage}
  \maketitle
  \begin{figure}[!htb]
    \centering
    \includegraphics[scale=0.30]{image.jpg}
    \caption{mechanical}
    \label{mechanical}
  \end{figure}

\end{titlepage}

\maketitle
  
\section{Introduction}

In classical mechanics is one of the two major sub-fields of mechanics, along with quantum mechanics. Classical mechanics is concerned with the set of physical laws describing the motion of bodies under the influence of a system of forces. The study of the motion of bodies is an ancient one, making classical mechanics one of the oldest and largest subjects in science, engineering and technology. It is also widely known as Newtonian mechanics.


\begin{titlepage}

\section{classical mechanics}
\

Classical mechanics describes the motion of macroscopic objects, from projectiles to parts of machinery, as well as astronomical objects, such as spacecraft, planets, stars, and galaxies. Within classical mechanics are fields of study that describe the behavior of solids, liquids and gases and other specific sub-topics. Classical mechanics also provides extremely accurate results as long as the domain of study is restricted to large objects and the speeds involved do not approach the speed of light. When the objects being examined are sufficiently small, it becomes necessary to introduce the other major sub-field of mechanics, quantum mechanics, which adjusts the laws of physics of macroscopic objects for the atomic nature of matter by including the wave–particle duality of atoms and molecules. When both quantum mechanics and classical mechanics cannot apply, such as at the quantum level with high speeds, quantum field theory (QFT) becomes applicable.


\subsection{Classical versus quantum mechanics}

Historically, classical mechanics came first, while quantum mechanics is a comparatively recent invention. Classical mechanics originated with Isaac Newton's laws of motion in Principal Mathematical; Quantum Mechanics was discovered in the early 20th century. Both are commonly held to constitute the most certain knowledge that exists about physical nature. Classical mechanics has especially often been viewed as a model for other so-called exact sciences. Essential in this respect is the relentless use of mathematics in theories, as well as the decisive role played by experiment in generating and testing them.



\subsection{General relativistic versus quantum}

Relativistic corrections are also needed for quantum mechanics, although general relativity has not been integrated. The two theories remain incompatible, a hurdle which must be overcome in developing a theory of everything.

\subsection{Description of the theory}

The following introduces the basic concepts of classical mechanics. For simplicity, it often models real-world objects as point particles (objects with negligible size). The motion of a point particle is characterized by a small number of parameters: its position, mass, and the forces applied to it. Each of these parameters is discussed in turn.

Classical mechanics uses common-sense notions of how matter and forces exist and interact. It assumes that matter and energy have definite, knowable attributes such as where an object is in space and its speed. Non-relativistic mechanics also assumes that forces act instantaneously (see also Action at a distance).
 
 {\bf {\large Velocity and speed}}

The velocity, or the rate of change of position with time, is defined as the derivative of the position with respect to time: V=ds/dt

When both objects are moving in the same direction, this equation can be simplified.

    
{\bf {\large Acceleration}}

 The acceleration, or rate of change of velocity, is the derivative of the velocity with respect to time (the second derivative of the position with respect to time):
 

\end{titlepage}
 

\section{conclusion}

Sub-disciplines in mechanics.
The following are two lists of various subjects that are studied in mechanics.

Note that there is also the "theory of fields" which constitutes a separate discipline in physics, formally treated as distinct from mechanics, whether classical fields or quantum fields. But in actual practice, subjects belonging to mechanics and fields are closely interwoven. Thus, for instance, forces that act on particles are frequently derived from fields (electromagnetic or gravitational), and particles generate fields by acting as sources. In fact, in quantum mechanics, particles themselves are fields, as described theoretically by the wave function.


\end{document}
